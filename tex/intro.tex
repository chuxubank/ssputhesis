%# -*- coding: utf-8-unix -*-
% !TEX program = xelatex
% !TEX root = ../thesis.tex
% !TEX encoding = UTF-8 Unicode
%%==================================================
%% chapter01.tex for SJTU Master Thesis
%%==================================================

%\bibliographystyle{sspu2}%[此处用于每章都生产参考文献]
\chapter{绪论}
\label{chap:intro}

\section{研究背景}

学生和白领在日常生活中经常遇到很多需要同时处理的任务,甚至需要同时参加几个大型的项目,
而这些任务的属性各不相同,有些需要高度集中的精力,有些则可以在短时间内完成,
那么如何正确而有效的管理人们日常生活中的大小任务和项目,并合理分配时间,便成为了一个重大议题。

\section{研究现状}
GTD \footnote{Get Things Done} 是由 David Allen 提出的对任务管理的一个行之有效的方法 \parencite{allen2015getting} ,
对于一般的任务管理,可以分为以下五个步骤:
\begin{enumerate}
	\item 捕捉:收集所有你在意的事情
	\item 明确:细化具体需要做的事情
	\item 组织:将任务放入对应的列表
	\item 思考:定期回顾任务
	\item 参与:自信地做任务
\end{enumerate}

项目管理作为工程专业的专业课程,已经被研究的很深入了,在此不做详细阐述。

对于任务管理,市面上已经出现了很多使用这一套 GTD 做法的应用:如 Things、OmniFocus 等,
而对于项目管理,则有 Project、OmniPlan 等比较成熟的面向企业的软件,
这些软件在各自的领域都表现的比较出色,但是始终缺少一个可以将任务管理和项目管理结合起来的面向个人用户的应用。

\section{研究目的和意义}
本课题旨在借助iOS端良好的生态和系统级的应用支持,将任务管理和项目管理相结合,为用户提供更为方便的任务管理系统和时间统计系统\parencite{hollemans2017ios},
同时加入精力管理,使得用户能更加清楚自己的精力状态,方便任务的安排\parencite{loehr2005power}。

在大学的学习生活的过程中,我发现一个良好的计划和将这些计划施行的行动力是平衡好学习和生活的“捷径”,
然而当我们在短时间内遇到很多复杂的事情和死线时难免出现不知所措的情况\parencite{mu2015time},
此时比起纸笔,借助于软件更能规划好将来的事情,并在适当的时候给予提醒,监督和统计个人规划\parencite{mcgonigal2011willpower}。 
这个软件不仅是一个工具,更是行为管理的方法。

\section{研究目标和内容}
\begin{itemize}
	\item GTD 方法理论
	
	研究GTD的基本思想,并将其运用到程序中。
	\item iOS 开发
	
	由于本科阶段并未学习关于iOS的开发,故需要系统学习iOS的开发过程,
	包括Apple \footnote{指苹果公司,下同} 提供的如 EventKit、Core ML等与本课题相关的框架。
	\item 任务管理
	
	研究任务的属性确定和组织排列方式等。
	\item 项目管理
	
	研究任务在项目中的组织方式和进度提示。
	\item 时间管理
	\item 精力管理
	
	通过研究 Jim Loehr 的精力管理理念,确定精力管理的方式。
	% \item 自然语言处理
	
	% 主要通过Apple提供的Core ML2机器学习框架进行自然语言处理,使用户添加任务更为快捷。
\end{itemize}

本课题的目标就是通过对以上内容的研究,开发一款帮助用户更方便的进行任务计划和时间管理的iOS端应用软件。

\section{开发环境简介}

\subsection{硬件开发环境}

\begin{enumerate}
	\item 处理器:Intel(R) Core(TM) i7-6700HQ
	\item 内存:8.0 GB
	\item 硬盘:256G SSD
	\item 显示:1920 $\times$ 1080 分辨率
\end{enumerate}

\subsection{软件开发环境}

\begin{enumerate}
	\item 操作系统:macOS Mojave 10.14.4
	\item 数据库系统:Core Data
	\item 开发语言:Swift 5
	\item 开发工具:Xcode 10.2.1
\end{enumerate}

\section{相关技术简介}
\subsection{Swift}
Swift 编程语言是 Apple 在 2014 年 WWDC \footnote{苹果公司的开发者大会} 上发布的新开发语言,可与 Objective-C 共同运行于 macOS 和 iOS 平台
,用于搭建基于苹果平台的应用程序,并于2015年12月4日开源\parencite{apple2018swift}。
\subsection{Core Data}
Core Data 是 Apple 开发的 Cocoa API 中的一部分,首次在iOS 3.0 时出现,
它允许按照实体-属性-值模型组织数据,并以XML,二进制文件或SQLite数据文件的格式将其序列化。
Core Data允许用户使用代表实体和实体间关系的高层对象来操作数据。
它也可以管理序列化的数据,提供对象生存期管理与对象图管理,包括存储。
Core Data直接与SQLite交互,从而避免开发者使用原本的SQL语句\parencite{com2018core}。