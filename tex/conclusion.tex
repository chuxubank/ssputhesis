%# -*- coding: utf-8-unix -*-
% !TEX program = xelatex
% !TEX root = ../thesis.tex
% !TEX encoding = UTF-8 Unicode

\chapter{结论}

通过几个月的不断学习和摸索,初步掌握了iOS的开发,并完成了
本任务计划和时间管理系统的开发。此次毕业设计中我对自己有了新的认识,也明白了很多道理。
首先是对于系统总体的把握,虽然此次开发的是时间管理系统,但我个人对时间的把控却十分不足,
虽然也有一些客观因素,但主观上的眼高手低是造成问题的关键,而本系统也旨在通过对
任务的拆分和项目中任务的管理,帮助和我类似的人群解决这一问题。
我曾在Android平台上开发过类似的系统,但在开发本系统时却无法进行任何利用,这在跨平台开发上是也是需要值得深思的问题。
这也说明了对于一些看似相似的知识,没有经过自己实际运用和实践,就无法保证知识的有效性。

在开发过程中我也遇到了很多问题,但苦于iOS的普及程度远不如Android高,很多问题无法得到及时的解决,
在学习资料上的缺失和Swift语言的新颖导致我始终无法系统而有效的学习iOS开发。
比如在学习Auto Layout进行界面UI的搭建时,我很难指定出我想要的约束条件,通过翻阅Apple的开发者文档,才逐渐掌握。
但经过我的不懈努力,通过查找大量英文资料和视频,我独立解决了大部分问题,在和同学的探讨中我也学习到了一些有效的解决办法。
本次系统运用了面向对象的设计思想,采用标准的MVC架构,很大程度上提高了代码的可重用性和可维护性。
开发过程是漫长而枯燥的,但是看到最终的成品,还是令人激动,这对我来说是一次绝好的将理论知识运用于实践的过程。
不仅是软件开发的相关理论,还有个人的价值观和对任务管理及生活的思考,这个作品也是包含着我的思想的。

本系统完成了任务计划和时间管理的基本功能,但由于个人的水平有限,还有很多地方值得改进。
如界面设计可以更具个性,对于任务之间的依赖关系可以用图表表示地更加清晰,
为添加任务、添加项目、添加事件增加统一的新增入口,更加方便用户操作,
增强精力管理,使用户能清晰的看到精力分配的时间,加强下一步推荐的算法,为用户提供更为合理的任务提醒。
总之,通过本次毕业设计,我对于时间管理的开发有了更加深入的理解。