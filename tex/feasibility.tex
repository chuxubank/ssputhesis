%# -*- coding: utf-8-unix -*-
% !TEX program = xelatex
% !TEX root = ../thesis.tex
% !TEX encoding = UTF-8 Unicode
\chapter{可行性分析}
本系统采用较为传统的瀑布模型进行软件开发,但该模型对于需求不经常变更的系统(如本系统)仍是十分有效的。
故先介绍本系统的定义阶段。
该章节主要讨论在当前的经济和技术条件下,能否在不亏损的情况下实现用户的需求。
对本课题的可行性分析从以下几个方面展开。
\section{经济可行性}
目前几乎所有的 GTD 类软件均为收费软件,部分免费软件则限制了一些使用功能,
而较为专业的项目管理软件则多为面向企业发售,且价格不菲。
除去 iOS 系统自带的提醒事项,针对个人的免费任务管理软件可谓少之又少。
本系统为独立开发者在业余时间开发,除去时间成本,若要进行发售则产生的费用只有 Apple 开发者账户的授权费用。
且本系统旨在培养生活习惯,可以进行长期使用,为用户剩下的时间成本将远高于软件费用,
因此,从长期来看,本系统的开发具备经济可行性。

\section{技术可行性}
本次开发的任务计划及时间管理系统主要针对个人用户,采用 Swift 为开发语言,
对传统的Objective-C语言进行了修补和更新,同时也是 Apple 今后的发展方向和主要开发语言。
由于Swift和Objective-C可以共存,故无需担心兼容性问题,但 Swift 面世时间相对较短,学习资料和开源库数量都较少。

数据库采用Apple 封装的Core Data,其强大的对象间关系和高度的抽象使得开发变得轻松了许多,
但同时数据源还包括Apple 系统中的EventKit框架,这也使得此系统只可能在Apple 开发的系统中使用。
因此本系统开发具备技术可行性。

\section{社会环境}
随着科技的进步,使用智能手机的人逐渐成为大多数,本系统基于iOS 这一手机端操作系统,
本系统操作界面与iOS系统应用操作方式和界面风格基本保持一致,使得上手和使用都较为轻松。
本系统的目标用户相对较为年轻,可以通过网络等新媒体进行推广。
同时本系统的开发将不会侵犯任何个人、集体、国家的利益,也不会违反国家的政策与法律。

综上所述,本任务计划和时间管理系统迎合了学生和白领的需求,
同时在经济、技术、社会环境方面都具备开发的可行性。由此可以得出本系统的开发是可行的结论。