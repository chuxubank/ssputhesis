%# -*- coding: utf-8-unix -*-
% !TEX program = xelatex
%%==================================================
%% thesis.tex
%%==================================================

\documentclass[bachelor, openany, oneside]{ssputhesis}

% 逐个导入参考文献数据库
\addbibresource{bib/thesis.bib}

%# -*- coding: utf-8-unix -*-
% !TEX program = xelatex
% !TEX root = ../thesis.tex
% !TEX encoding = UTF-8 Unicode
%TC:ignore
\title{基于Latex的论文模版}
\author{某\quad{}某}
\advisor{某某老师}
% \coadvisor{某某教授}
\defenddate{2019年5月8日}
\coursename{某某课程}
\school{上海第二工业大学}
\institute{某某学部}
\studentnumber{12345678901}
\cnacademicdegree{工学硕士}
\major{某某专业}
\class{某某班级}
\entrancetime{某某级}
\keywords{SSPU;Latex;论文}

\englishtitle{iOS-Based Task Plan and Time Management System}
\englishauthor{\textsc{Mo Mo}}
\englishadvisor{Prof. \textsc{Mou Mou}}
% \englishcoadvisor{Prof. \textsc{Uom Uom}}
\englishschool{Shanghai Jiao Tong University}
\englishinstitute{\textsc{Depart of XXX, School of XXX} \\
  \textsc{Shanghai Jiao Tong University} \\
  \textsc{Shanghai, P.R.China}}
\englishinstitutemaster{Depart of XXX, \\ School of XXX}
\englishmajor{A Very Important Major}
\englishdate{Dec. 17th, 2014}
\enacademicdegree{Master of Engineering}
\englishstudentid{0010900990}
\englishkeywords{SSPU, Latex, Thesis}
%TC:endignore
  % NOTE: the enclosed commands must be executed in preamble

\begin{document}

% 无编号内容:中英文论文封面、授权页
\maketitle
% 使用 @ 字符
\makeatletter

\ifsspu@coursepaper
% 摘要 部分课程论文需要,可以自行选择添加或者去除
%# -*- coding: utf-8-unix -*-
% !TEX program = xelatex
% !TEX root = ../thesis.tex
% !TEX encoding = UTF-8 Unicode
%%==================================================
%% abstract.tex for SJTU Master Thesis
%%==================================================

\begin{abstract}

任务计划和时间管理和每个人的生活和工作都息息相关,
随着社会生活节奏的加快,人们逐渐被各种各样的任务和项目打乱原本的节奏,
无法平衡生活和工作,难以记录自己花费的时间,难以分配自己的精力。
本文主要介绍了一个基于iOS的任务计划和时间管理软件的设计与实现。
该系统旨在通过软件的方式,结合GTD和精力管理的思想,帮助人们规划任务,管理时间和精力。

本系统主要分为今日计划、任务管理、项目管理、精力管理、行为管理五个模块。
今日计划实现对于今天的任务和事件的安排的管理,任务管理通过对任务进行详尽的设置和项目分类,对任务进一步细化,
项目管理对任务之间的依赖关系进行处理和安排,精力管理可以展示用户对于每个任务花费的精力,行为管理是任务管理的进一步细化,
对用户的行为进行记录,并进行展示。

本系统采用Swift 5语言开发,使用方便,界面友好,能够帮助用户合理规划和安排时间。


\end{abstract}

\begin{englishabstract}

Mission planning and time management are closely related to everyone’s life and work.
As the pace of social life accelerates, people are gradually disrupted by the various tasks and projects.
Unable to balance life and work, it is difficult to record the time spent, and it is difficult to allocate your energy.
This paper mainly introduces the design and implementation of an iOS-based task plan and time management software.
The system is designed to help people plan tasks and manage time and effort through software, combined with GTD and energy management ideas.

The system is mainly divided into five modules: planning, task management, project management, energy management and behavior management.
Today plans to manage the scheduling of today's tasks and events. Task management further refines the tasks by detailed setting and classification of tasks.
Project management processes and arranges the dependencies between tasks. Energy management can show the user's energy spent on each task. Behavior management is a further refinement of task management.
Record and display the user's behavior.

This system is developed in Swift 5 language. It is easy to use and friendly. It can help users plan and schedule time reasonably.

\end{englishabstract}



% 目录 部分课程论文需要,可以自行选择添加或者去除
\tableofcontents

\else
  \ifsspu@submit\relax
    \includepdf{pdf/original.pdf}
    \cleardoublepage
    \includepdf{pdf/authorization.pdf}
    \cleardoublepage
  \else
    \ifsspu@review\relax
    % exclude the original claim and authorization
    \else
	  \makeDeclareOriginal
	%   授权页
    %   \makeDeclareAuthorization
    \fi
  \fi
  \frontmatter % 使用罗马数字对前言编号

  % 摘要
  %# -*- coding: utf-8-unix -*-
% !TEX program = xelatex
% !TEX root = ../thesis.tex
% !TEX encoding = UTF-8 Unicode
%%==================================================
%% abstract.tex for SJTU Master Thesis
%%==================================================

\begin{abstract}

任务计划和时间管理和每个人的生活和工作都息息相关,
随着社会生活节奏的加快,人们逐渐被各种各样的任务和项目打乱原本的节奏,
无法平衡生活和工作,难以记录自己花费的时间,难以分配自己的精力。
本文主要介绍了一个基于iOS的任务计划和时间管理软件的设计与实现。
该系统旨在通过软件的方式,结合GTD和精力管理的思想,帮助人们规划任务,管理时间和精力。

本系统主要分为今日计划、任务管理、项目管理、精力管理、行为管理五个模块。
今日计划实现对于今天的任务和事件的安排的管理,任务管理通过对任务进行详尽的设置和项目分类,对任务进一步细化,
项目管理对任务之间的依赖关系进行处理和安排,精力管理可以展示用户对于每个任务花费的精力,行为管理是任务管理的进一步细化,
对用户的行为进行记录,并进行展示。

本系统采用Swift 5语言开发,使用方便,界面友好,能够帮助用户合理规划和安排时间。


\end{abstract}

\begin{englishabstract}

Mission planning and time management are closely related to everyone’s life and work.
As the pace of social life accelerates, people are gradually disrupted by the various tasks and projects.
Unable to balance life and work, it is difficult to record the time spent, and it is difficult to allocate your energy.
This paper mainly introduces the design and implementation of an iOS-based task plan and time management software.
The system is designed to help people plan tasks and manage time and effort through software, combined with GTD and energy management ideas.

The system is mainly divided into five modules: planning, task management, project management, energy management and behavior management.
Today plans to manage the scheduling of today's tasks and events. Task management further refines the tasks by detailed setting and classification of tasks.
Project management processes and arranges the dependencies between tasks. Energy management can show the user's energy spent on each task. Behavior management is a further refinement of task management.
Record and display the user's behavior.

This system is developed in Swift 5 language. It is easy to use and friendly. It can help users plan and schedule time reasonably.

\end{englishabstract}



  % 目录、插图目录、表格目录
  \tableofcontents
  \listoffigures
  \addcontentsline{toc}{chapter}{\listfigurename}     % 将插图目录加入全文目录
  \listoftables
  \addcontentsline{toc}{chapter}{\listtablename}      % 将表格目录加入全文目录
  \listofalgorithms
  \addcontentsline{toc}{chapter}{\listalgorithmname}  % 将算法目录加入全文目录

  \include{tex/symbol} % 主要符号、缩略词对照表
\fi

\makeatother
\mainmatter % 使用阿拉伯数字对正文编号

% 正文内容
%# -*- coding: utf-8-unix -*-
% !TEX program = xelatex
% !TEX root = ../thesis.tex
% !TEX encoding = UTF-8 Unicode
%%==================================================
%% chapter01.tex for SJTU Master Thesis
%%==================================================

%\bibliographystyle{sspu2}%[此处用于每章都生产参考文献]
\chapter{绪论}
\label{chap:intro}

\section{研究背景}

学生和白领在日常生活中经常遇到很多需要同时处理的任务,甚至需要同时参加几个大型的项目,
而这些任务的属性各不相同,有些需要高度集中的精力,有些则可以在短时间内完成,
那么如何正确而有效的管理人们日常生活中的大小任务和项目,并合理分配时间,便成为了一个重大议题。

\section{研究现状}
GTD \footnote{Get Things Done} 是由 David Allen 提出的对任务管理的一个行之有效的方法 \parencite{allen2015getting} ,
对于一般的任务管理,可以分为以下五个步骤:
\begin{enumerate}
	\item 捕捉:收集所有你在意的事情
	\item 明确:细化具体需要做的事情
	\item 组织:将任务放入对应的列表
	\item 思考:定期回顾任务
	\item 参与:自信地做任务
\end{enumerate}

项目管理作为工程专业的专业课程,已经被研究的很深入了,在此不做详细阐述。

对于任务管理,市面上已经出现了很多使用这一套 GTD 做法的应用:如 Things、OmniFocus 等,
而对于项目管理,则有 Project、OmniPlan 等比较成熟的面向企业的软件,
这些软件在各自的领域都表现的比较出色,但是始终缺少一个可以将任务管理和项目管理结合起来的面向个人用户的应用\parencite{alvarez2019time}。

\section{研究目的和意义}
本课题旨在借助iOS端良好的生态和系统级的应用支持,将任务管理和项目管理相结合,为用户提供更为方便的任务管理系统和时间统计系统\parencite{hollemans2017ios},
同时加入精力管理,使得用户能更加清楚自己的精力状态,方便任务的安排\parencite{loehr2005power}。

在大学的学习生活的过程中,我发现一个良好的计划和将这些计划施行的行动力是平衡好学习和生活的“捷径”,
然而当我们在短时间内遇到很多复杂的事情和死线时难免出现不知所措的情况\parencite{mu2015time},
此时比起纸笔,借助于软件更能规划好将来的事情,并在适当的时候给予提醒,监督和统计个人规划\parencite{mcgonigal2011willpower}。 
这个软件不仅是一个工具,更是行为管理的方法。

\section{研究目标和内容}
\begin{itemize}
	\item GTD 方法理论
	
	研究GTD的基本思想,并将其运用到程序中。
	\item iOS 开发
	
	由于本科阶段并未学习关于iOS的开发,故需要系统学习iOS的开发过程,
	包括Apple \footnote{指苹果公司,下同} 提供的如 EventKit、Core ML等与本课题相关的框架。
	\item 任务管理
	
	研究任务的属性确定和组织排列方式等。
	\item 项目管理
	
	研究任务在项目中的组织方式和进度提示。
	\item 时间管理
	\item 精力管理
	
	通过研究 Jim Loehr 的精力管理理念,确定精力管理的方式。
	% \item 自然语言处理
	
	% 主要通过Apple提供的Core ML2机器学习框架进行自然语言处理,使用户添加任务更为快捷。
\end{itemize}

本课题的目标就是通过对以上内容的研究,开发一款帮助用户更方便的进行任务计划和时间管理的iOS端应用软件。

\section{开发环境简介}

\subsection{硬件开发环境}

\begin{enumerate}
	\item 处理器:Intel(R) Core(TM) i7-6700HQ
	\item 内存:8.0 GB
	\item 硬盘:256G SSD
	\item 显示:1920 $\times$ 1080 分辨率
\end{enumerate}

\subsection{软件开发环境}

\begin{enumerate}
	\item 操作系统:macOS Mojave 10.14.4
	\item 数据库系统:Core Data
	\item 开发语言:Swift 5
	\item 开发工具:Xcode 10.2.1
\end{enumerate}

\section{相关技术简介}
\subsection{Swift}
Swift 编程语言是 Apple 在 2014 年 WWDC \footnote{苹果公司的开发者大会} 上发布的新开发语言,可与 Objective-C 共同运行于 macOS 和 iOS 平台
,用于搭建基于苹果平台的应用程序,并于2015年12月4日开源\parencite{apple2018swift}。
\subsection{Core Data}
Core Data 是 Apple 开发的 Cocoa API 中的一部分,首次在iOS 3.0 时出现,
它允许按照实体-属性-值模型组织数据,并以XML,二进制文件或SQLite数据文件的格式将其序列化。
Core Data允许用户使用代表实体和实体间关系的高层对象来操作数据。
它也可以管理序列化的数据,提供对象生存期管理与对象图管理,包括存储。
Core Data直接与SQLite交互,从而避免开发者使用原本的SQL语句\parencite{com2018core}。
\include{tex/example}
\include{tex/faq}
\include{tex/summary}

\appendix % 使用英文字母对附录编号

% 附录内容,本科学位论文可以用翻译的文献替代。
\include{tex/app_setup}
\include{tex/app_eq}
\include{tex/app_cjk}
\include{tex/app_log}

\backmatter % 文后无编号部分

% 参考资料
\printbibliography[heading=bibintoc]

% 致谢、发表论文、申请专利、参与项目、简历
% 用于盲审的论文需隐去致谢、发表论文、申请专利、参与的项目
\makeatletter

\ifsspu@coursepaper
\else

  % "研究生学位论文送盲审印刷格式的统一要求"
  % http://www.gs.sspu.edu.cn/inform/3/2015/20151120_123928_738.htm

  % 盲审删去删去致谢页
  \ifsspu@review\relax\else
    %# -*- coding: utf-8-unix -*-
% !TEX program = xelatex
% !TEX root = ../thesis.tex
% !TEX encoding = UTF-8 Unicode
%TC:ignore
\begin{thanks}

大学四年的学习生活即将结束,在本次毕业设计完成之际,
我要特别感谢我的指导老师李丽萍老师,老师秉承认真负责的工作态度,
在整个毕业设计过程中定时督促、检查我的系统完成情况,
从中指出我的不足之处,在我遇到困难时给予耐心的指导。
我能够完成系统的设计、编码和论文,还要感谢同学们的无私帮助。
非常感谢他们能够抽出时间,帮助我发现问题,解决问题,给出了很多有用的建议和补充。
最后,我由衷的感谢所有教导过我的老师们,
你们严谨的治学风格,让我很好的学习了专业知识;
你们独特的人格魅力,感染教导我为人处事。
还要感谢我的母校——上海第二工业大学四年来对我的精心培养,
让我能够变的更加出色,让我有自信成为社会的有用之才。

\end{thanks}
%TC:endignore
         % 致谢
  \fi

  \ifsspu@bachelor
    % 学士学位论文要求在最后有一个英文大摘要,单独编页码
    \include{tex/end_english_abstract}
  \else
    % 盲审论文中,发表学术论文及参与科研情况等仅以第几作者注明即可,不要出现作者或他人姓名
    \ifsspu@review\relax
      \include{tex/pubreview}
      \include{tex/projectsreview}
    \else
      \include{tex/pub}       % 发表论文
      \include{tex/projects}  % 参与的项目
      % \include{tex/patents}   % 申请专利
      \include{tex/resume}    % 个人简历
    \fi
  \fi
\fi

\makeatother

\end{document}
